\documentclass[a4paper,ngerman]{scrartcl}
\usepackage[utf8]{inputenc}
\usepackage[T1]{fontenc}
\usepackage{babel}
\usepackage{textcomp}
\usepackage{graphicx}
\usepackage{nameref}
\usepackage{smartref}
\addtoreflist{section}
%% Section numbering starts with $
%\renewcommand*\thesection{\S~\arabic{section}}
%% Hyperref allows us to have clickable links in the tables of content (i.e., our todo list)
%\usepackage{hyperref}
%% Use todo notes
%\usepackage{todonotes}
%% Allow us to reuse title info later (i.e., reuse the date as \MyDate)
\usepackage{authoraftertitle}
\setlength{\parindent}{0em}
\begin{document}

Die Mitgliederversammlung des Vereins Squareroots e.V. hat am [Datum] folgende Beitragsordnung beschlossen:
\\
\\
Beitragsordnung des [Name des Vereins] e.V.

\begin{enumerate}
\item Alle Vereinsmitglieder zahlen einen Mitgliedsbeitrag. Der Mitgliedsbeitrag wird [Zahlungsmodus: monatlich / quartalsweise / halbjährlich / jährlich] erhoben. Ehrenmitglieder sind von der Beitragszahlung befreit.
\item Die Beiträge werden jeweils zum ersten Werktag [je nach Zahlungsmodus: eines jeden Monats / im ersten Monat des laufenden Quartals / Halbjahres / Jahres] eingezogen. Das Mitglied erteilt dem Verein hierfür ein SEPA-­‐LastschriRmandat.
\item Der [je nach Zahlungsmodus: monatliche /vierteljährliche / halbjährliche / jährliche] Beitrag beträgt:
\begin{enumerate}
\item Für Erwachsene (ab dem vollendeten 18. Lebensjahr) [Betrag in Euro]
\item Für Jugendliche (ab dem vollendeten 14. Lebensjahr) [Betrag in Euro]
\item Für Kinder (bis zum vollendeten 14. Lebensjahr) [Betrag in Euro]
\item Familien (2 Erwachsene und bis zu 3 Kinder / Jugendliche] [Betrag in Euro]
\item Ehepaare, eingetragene LebensgemeinschaRen ohne Kinder [Betrag in Euro]
\item Senioren (ab dem vollendeten 65. Lebensjahr) [Betrag in Euro]
\end{enumerate}
\item Der Verein erhebt eine Aufnahmegebühr von [Betrag in Euro], die nach Aufnahme in den Verein fällig wird und im LastschriRverfahren eingezogen wird.
\item Es können Umlagen und / oder Sachleistungen von den Mitgliedern erhoben werden. Die Erhebung von Umlagen und / oder Sachleistungen muss von der Mitgliederversammlung beschlossen werden.
\item Alle aktiven Mitglieder ab dem vollendeten 16. Lebensjahr und bis zum vollendeten 60. Lebensjahr müssen jährlich [Anzahl] Stunden Arbeit zum Erhalt und / oder zur Pflege der vereinseigenen Einrichtungen und Anlagen erbringen. Wird die Anzahl der Arbeitsstunden nicht erfüllt, erhebt der Verein pro nicht geleisteter Stunde [Betrag in Euro]. Der eventuell fällig werdende Betrag wird per Lastschrifteinzug im Monat abgebucht, der auf den Monat folgt, in dem das Mitglied über die Abrechnung der Stunden informiert wurde.
\item Diese Beitragsordnung kann bei Notwendigkeit vom Vorstand per Beschluss geändert werden. Der Vorstand hat Änderungsbeschlüsse bezüglich dieser Satzung in der nächsten Mitgliederversammlung vorzulegen.
\end{enumerate}


Beschlossen am [Datum]
\end{document}
